\chapter{結論}
\label{chap:conclusion}

\section{まとめ}
\label{section:conclusion_sum}

本論文は、運動不足で生活習慣病になる患者の中でも、糖尿病になる患者に焦点を当て、その治療の負担を軽くするための提案としてのインスリン打ち忘れ通知システムについて述べてきた。
そして、食前のインスリンを摂取し忘れてしまい、それに気づかないまま食事を行ってしまう患者が、それを知る術がないという問題を解決することを目標としていた。
そしてその解決手法として、インスリン摂取時間と食事時間の2つのイベントが起こった時間を、実装したデバイスにより検知し、比較することでインスリンの摂取忘れを検知し通知するシステムを提案した。
インスリン摂取検知デバイスはESP32とタクトスイッチを用いて実装し、食事検知デバイスはADXL345とRaspberry Pi 4を用いて実装し、食事検知のアルゴリズムは四人の食事データをもとに作成した。
インスリン摂取検知デバイスと食事検知デバイスがそれぞれ期待通り動作することを確認するため、別々に評価を行い、インスリン摂取検知デバイスは期待通り動作したことを確認した。
そして、食事検知に関しては、被験者三人とも食事をしている時間に食事が検知できることがわかった。
最後に、これらの二つを合わせて評価を行い、食前にインスリン摂取が検知されず、食事検知が行われた場合、被験者に対して音声通知が伝達されることが確認できた。
以上の評価をもって、本論文の提案手法は、対象は限定的であったため、部分的ではあるが、解決したかった問題を解決できたと言える。

\section{今後の展望}

今回のアルゴリズムは固定の閾値を指定形で食事検知を行っていたが、これよりさらにデータを集めることで、将来的に機械学習の利用も考えられ、それによる精度向上を期待できる。
また、加速度に加えて音声検知を行うことで、皿が置かれた時や、フォーク、ナイフを使っての食事など、精度に加えて対応可能になる食事場面が多くなる可能性が見込める。
また、スマホアプリ化してアプリにアルゴリズムを組み込むことで食事検知デバイスとして携帯可能になり、外食時でも利用可能な携帯することも可能性として考えられる。
インスリンペンに装着するデバイスに関して、今回はESP32に付けた配線をそのままつける形で実装になってしまったが、デバイス自体の小型化、電池駆動式に変更することで、外食時の携帯にも対応可能にさせることができる。
