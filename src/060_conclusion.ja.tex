\chapter{結論}
\label{chap:conclusion}

\section{まとめと考察}
\label{section:conclusion_sum}

一人暮らしの糖尿病患者自宅で食事をする患者は、他人に頼ることなく自己管理で治療を続ける必要がある。
そこで本論文では、食前のインスリンを摂取し忘れてしまい、それに気づかないまま食事を行ってしまう糖尿病患者が、それを知る術がないという問題を解決することを目指した。
インスリン摂取時間と食事検知時間を取得するデバイスを実装し、20代男性3名と20代女性1名でこれの評価を行った結果、食事検知で64\%の精度であった。
また、食事以外の行動では、書き物、お茶を飲む、マウスを使ってネットサーフィンが12.5\%、20\%、20\%という誤検知率と低くかった。
通知の評価では、インスリンを忘れて通知を行うべきシナリオで通知があった割合は66.67\%となったが、食事検知が行われた場合の通知成功率は100\%であった。
結果としては、パスタなど、食器を持ち上げる行為を誘引しにくい食事内容に関しては検知ができないなど、食事検知が可能な対象は限定的なものとなった。

\section{今後の展望}

今回のアルゴリズムは固定の閾値を指定形で食事検知を行っていたが、これよりさらにデータを集めることで、将来的に機械学習の利用も考えられ、それによる精度向上を期待できる。
また、加速度に加えて音声検知を行うことで、皿が置かれた時や、フォーク、ナイフを使っての食事など、精度に加えて対応可能になる食事場面が多くなる可能性が見込める。
また、スマホアプリ化してアプリにアルゴリズムを組み込むことで食事検知デバイスとして携帯可能になり、外食時でも利用可能な携帯することも可能性として考えられる。
インスリンペンに装着するデバイスに関して、今回はESP32に付けた配線をそのまま付ける形で実装になってしまったが、デバイス自体の小型化、電池駆動式に変更することで、外食時の携帯にも対応可能にさせることができる。
