\begin{acknowledgment}

  本論文の執筆にあたり、ご指導いただきました慶應義塾大学環境情報学部教授 中村修博士、同学部教授 ロドニー・バンミータ博士、
  同学部教授 楠本博之教授、同学部准教授 植原啓介准教授、同研究科特任准教授 松谷健史博士、 東京大学特任講師 空閑洋平博士に感謝いたします。
  特に松谷健史博士には、研究を進める上で何度もご指導や助言をいただき、非常にお世話になりました。この場を借りて重ねて御礼申し上げます。
  本当にありがとうございました。

  また、一年生の秋に入った本研究室で共に学生生活を過ごした先輩、同期、後輩の皆様には様々な刺激をいただき、感謝いたします。

  そして、休学期間中の留学で出会った人全てに感謝いたします。2年間のアメリカ生活の中で、文化の違い、生活様式の違い、そして考え方の違いから
  本当に多くの刺激を受け、人間として大きく成長できたと実感しています。そして、自分が参加した42シリコンバレー校の設立、運営に関わった全ての方に感謝いたします。
  特に、一緒に参加し、毎日の苦しくも楽しい日々を共にしたRajeevan Sathiadevan氏、Matteo Mateo氏、Gokulan Gnanendran氏、Cameron Wood氏、Rail Sharipov氏、
  Dmitry Kotov氏、Yosef Serkez氏、Juo-Wei Chen氏には、多くの刺激を受け、想像力を持って他者への理解を示すことの大切さを学びました。
  この場を借りて深く感謝いたします。

  本研究を進めるにあたって、忙しい中時間を割いて研究の実験に協力してくれた、宇佐美吉高氏、牛嶋将貴氏、滝本将大氏、柏木秀治氏、
  橋本拓実氏、山田結衣里氏に深く感謝いたします。また、学生生活を共にし、励ましの言葉をくれた石井晃太郎氏に感謝いたします。

  私が所属していたサークルであるTeamSwearの皆様に感謝いたします。大変お世話になりました。

  そして最後に、これまで私の学生生活を支え、あらゆる活動を支援してくれた家族に感謝いたします。
  今まで本当に多くの方に支えられ、様々な経験を積むことができました。ありがとうございました。
  以上を以って本論文の謝辞とさせていただきます。

\end{acknowledgment}
