\chapter{評価}
\label{chap:evaluation}

本章では、本研究で実装したインスリンデバイスと食事検知デバイスのそれぞれの性能を評価する。
そしてその二つの評価結果について論じた後、二つを合わせてインスリン打ち忘れ通知に関する評価とその結果をもとに考察を述べる。

\section{インスリンデバイス}
インスリンデバイスの性能について評価をし、考察する。

\subsection{評価環境}
インスリンペンデバイスを評価するための実験環境について述べる。

\subsection{誤検知率}
誤検知率がどれくらいになったのかを述べる。

\subsection{検知時間と実摂取時間の差異}
検知時間と実際に摂取した時間にどれくらい誤差があるのかを見る。

\section{食事検知}
食事検知の性能について評価をし、考察する。

\subsection{評価環境}
食事検知を評価するための実験環境について述べる。

\subsection{誤検知率}
誤検知率がどれくらいになったのかを述べる。

\subsection{間時間と実摂取時間の差異}
検知時間と実際に食事をとった時間にどれくらい誤差があるのかを見る。

\section{インスリン摂取忘れ通知}
上記2つの評価をもとに、インスリン摂取忘れ通知が達成できるかどうかについて評価し、考察する。

\subsection{評価環境}
インスリン摂取忘れ通知の評価をするための実験環境について述べる。

\subsection{シナリオ}
2つのシナリオを用意して、得られた結果をここに載せる。(テーブル)


\subsection{考察}

上記に出した結果をもとに、考察を述べる。

