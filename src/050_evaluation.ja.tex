\chapter{評価}
\label{chap:evaluation}

本章では、本論文で実装した通知システムの評価を行う。そのために、まずはインスリン摂取検知デバイスと食事検知デバイスを個別に評価する。
そして、最後に2つを組み合わせて全体の通知システムを評価し、結果を受けて、本提案手法が本論文であげた解決したい問題を解決できるものであるか考察していく。

\section{インスリン摂取検知デバイス}
本節では、インスリン注射器に見立てたペンに装着されたインスリン摂取検知デバイスの評価を行う。

\subsection{評価方法}
第\ref{chap:diabetes}章でも述べた通り、糖尿病患者のインスリン治療では、ペン型のインスリン注射器を用いて必要な量のインスリンを服用する。
そして、本論文の評価では、この動作が再現でき、その動作を以ってインスリン摂取とすれば良い。
よって、今回はインスリンを実際に摂取するのではなく擬似的に\textbf{ペンを握り、5秒間スイッチを押して静止する}\cite{how_to_inject_insulin_1}\cite{how_to_inject_insulin_2}
という動作を行うことでインスリン摂取を行ったとみなした。
今回は、これを10回試行し、試行間隔は約10秒で行い、webサーバーを通してその時間が正確に記録されているかどうかを確認した。
試行間隔は10秒、そして検知のための静止時間は5秒であるため、検知時間の間隔は約15秒になっていることが期待される。

\subsection{結果}

テーブル\ref{tb:insulin_detection_time}が、インスリン摂取検知デバイスを使用して上記の動作を10回試行した結果である。
\textbf{\textit{injected\_time}}が、インスリン摂取検知デバイスから送られた検知時間であり、\textbf{\textit{created\_at}}がmysqlに保存された時間である。
この結果から、検知された時間が正確かつ即時的に保存されていることがわかる。

\newpage

\begin{table}[htbp]
  \caption{インスリン摂取検知デバイスによってリクエストされ、保存されたデータ}
  \label{tb:insulin_detection_time}
  \begin{center}
    \begin{tabular}{|c||l|l|}
      \hline
      ID & injected\_time & created\_at \\\hline
      1 & 2021-01-24 00:24:58 & 2021-01-24 00:24:58 \\\hline
      2 & 2021-01-24 00:25:13 & 2021-01-24 00:25:13 \\\hline
      3 & 2021-01-24 00:25:28 & 2021-01-24 00:25:28 \\\hline
      4 & 2021-01-24 00:25:44 & 2021-01-24 00:25:44 \\\hline
      5 & 2021-01-24 00:25:59 & 2021-01-24 00:25:59 \\\hline
      6 & 2021-01-24 00:26:13 & 2021-01-24 00:26:13 \\\hline
      7 & 2021-01-24 00:26:28 & 2021-01-24 00:26:29 \\\hline
      8 & 2021-01-24 00:26:44 & 2021-01-24 00:26:44 \\\hline
      9 & 2021-01-24 00:27:00 & 2021-01-24 00:27:00 \\\hline
      10 & 2021-01-24 00:27:15 & 2021-01-24 00:27:15 \\\hline
    \end{tabular}
  \end{center}
\end{table}

\subsection{考察}

結果から、10回全て約15秒間隔でインスリン摂取検知時間が取得できており、インスリン摂取検知デバイスは期待した通り動作していることが確認できた。

\section{食事検知デバイス}

本節では、食事検知デバイスの評価を行う。

\subsection{評価手法}
作成した食事検知デバイスを食卓上に置き、場面別で様々な動作を行い、食事の場面では\textbf{食事として検知}、それ以外の場合は\textbf{食事ではない}として、検知しないことを期待する。

今回評価を行った卓上での行動は以下の通りである。

\begin{enumerate}
  \item 食事をする
  \item パソコンで作業する
  \item ものを書く
  \item 机を拭く
\end{enumerate}

1の\textbf{食事をする}では、図\ref{}のような1主3菜のメニューを被験者に食べてもらった。図\ref{}が実際の実験の様子である。
2の\textbf{パソコンで作業する}では、タイピングが机に与える主な振動となるため、寿司打\cite{sushida}を5分間行ってもらった。
3の\textbf{ものを書く}では、ペンで何かを書く動作、消しゴムで消す動作、消しカスを払う動作、の3つが主な振動源となるためその三つを満たせるよう数学の問題を解く、絵を描くなどを行ってもらった。
4の\textbf{机を拭く}では、ペーパータオルで机を拭いてもらった。

これらの動作の際に起こる振動に対しても食事検知デバイスで正確に食事のみ検知できるかを評価する。

\subsection{結果}

本論文の評価では、20代男性2人、20代女性1人、合計3人の被験者に協力してもらい、食事の検知実験とその他の動作の検知実験を行った。
テーブル\ref{tb:meal_detection_result}が、それぞれの被験者の食事時間と、食事検知デバイスが食事を検知した時間である。

\begin{table}[htbp]
  \caption{それぞれの被験者の食事時間と食事検知時間}
  \label{tb:meal_detection_result}
  \begin{center}
    \begin{tabular}{|c||c|c|}
      \hline
      被験者  & 食事時間 & 食事検知時間 \\
      \hline\hline
      被験者A & 2021-01-24 18:53:35 - 19:01:12 & 2021-01-24 18:55:28.264315 \\\hline
      被験者B & 2021-01-24 19:04:05 - 19:11:10 & 2021-01-24 19:04:28.447520 \\\hline
      被験者C & 2021-01-21 12:46:51 - 12:58:30 & 2021-01-21 12:49:10.842095 \\\hline
    \end{tabular}
  \end{center}
\end{table}

平均で、食事を開始してから約1分半で検知に至り、食事をしている間に検知できたことがわかった。

テーブル\ref{tb:others_detection_time}が、その他の、パソコンでの作業、ものを書く、机を拭く動作をしてもらった時間であり、
テーブル\ref{tb:others_detection_result}その食事検知デバイスの検知結果である。

\begin{table}[htbp]
  \caption{それぞれの被験者のその他の動作を行なった時間}
  \label{tb:others_detection_time}
  \begin{center}
    \begin{tabular}{|c||c|c|c|}
      \hline
       & パソコンでの作業 & 書き物 & 机を拭く \\
      \hline\hline
      被験者A & 2021-01-24 19:39 - 19:47 & 2021-01-24 19:59 - 20:13 & 2021-01-24 20:31 - 20:32 \\\hline
      被験者B & 2021-01-24 19:50 - 19:57 & 2021-01-24 20:23 - 20:29 & 2021-01-24 20:38 - 20:38  \\\hline
      被験者C & 2021-01-26 16:36 - 16:43 & 2021-01-26 16:44 - 16:52 & 2021-01-26 16:53 - 16:53 \\\hline
    \end{tabular}
  \end{center}
\end{table}

\begin{table}[htbp]
  \caption{それぞれの被験者のその他の動作を行なった時の検知結果}
  \label{tb:others_detection_result}
  \begin{center}
    \begin{tabular}{|c||c|c|c|}
      \hline
       & パソコンでの作業 & 書き物 & 机を拭く \\
      \hline\hline
      被験者A & 検知なし & 検知なし & 検知なし \\\hline
      被験者B & 検知なし & 検知なし & 検知なし \\\hline
      被験者C & 検知なし & 検知なし & 検知なし \\\hline
    \end{tabular}
  \end{center}
\end{table}

\subsection{考察}

食事検知では、実際の食事開始時間と食事を検知できた時間に少々誤差はあったが、食事中にそのイベントを検知することが達成でき、デバイスが正常に動作していることがわかった。
また、その他の動作に関して、今回対象とした動作自体が限定的ではあったものの、これらの動作に関しては食事検知デバイスは反応せず、その機能の有効性を示したと言える。

\section{インスリン摂取忘れ通知}
ここまでで評価した二つのモジュールを使用して、最終的にインスリン摂取を忘れている際に通知が行えるかどうかを評価する。
本節では、食事検知デバイスの評価の際に参加してもらった20代女性に再度協力してもらい、評価を行なった。

\subsection{評価手法}
インスリン摂取忘れ通知の評価をするための実験環境について述べる。
被験者は手元に記録用媒体を準備し、シナリオ別にアラート音声を聞いたかどうか、またどのタイミングで聞いたかを記録する。

\subsubsection{シナリオ}

本研究では、以下の二つを実験シナリオとして実施した。

\begin{enumerate}
  \item 被験者は、インスリン摂取検知デバイスを使用して、食前にインスリン投与と同じアクション(インスリン摂取検知デバイスのスイッチを押して5秒静止)を行う。
  投与のアクションを終えたのち、食事を開始する。
  \begin{description}
    \item[期待する結果]\mbox{}\\
      インスリン注射器に装着されたデバイスが送信した投与時間は食事検知時間よりも前で、インスリンが正常に投与されたものとして判定。アラートの音声は発されない。
  \end{description}
  \item 被験者は、インスリン検知デバイスを使用することなく、食事を始める。
  \begin{description}
    \item[期待する結果]\mbox{}\\
      食事検知時に直近30分で、インスリンが正常に投与されなかったものとして判定。アラートの音声が発される。
  \end{description}
\end{enumerate}

\subsection{結果}
テーブル\ref{tb:result_scenarios}が、評価結果である。
期待した通り、シナリオ1では、食事検知が行われたものの、インスリンが食事前に摂取されていたため、通知は行われなかった。
そしてシナリオ2では、食事検知が行われ、食事前のインスリン摂取が行われていなかったため、期待した通り通知が行われた。

\begin{table}[htbp]
  \caption{シナリオ別の評価結果}
  \label{tb:result_scenarios}
  \begin{center}
    \begin{tabular}{|c||c|}
      \hline
      シナリオ No.  & 結果 \\
      \hline\hline
      1  & 通知なし \\\hline
      2 & 通知あり \\\hline
    \end{tabular}
  \end{center}
\end{table}

\begin{table}[htbp]
  \caption{シナリオ1のインスリン摂取時間}
  \label{tb:scenario_1_insulin}
  \begin{center}
    \begin{tabular}{|c||c|c|}
      \hline
      ID  & injected\_time & created\_at \\
      \hline\hline
      1 &  &  \\\hline
    \end{tabular}
  \end{center}
\end{table}

\begin{table}[htbp]
  \caption{シナリオ1の食事時間}
  \label{tb:scenario_2_meal}
  \begin{center}
    \begin{tabular}{|c||c|}
      \hline
        & 食事時間 \\
      \hline\hline
      被験者C & \\\hline
    \end{tabular}
  \end{center}
\end{table}

\begin{table}[htbp]
  \caption{シナリオ1の食事検知時間}
  \label{tb:scenario_1_meal}
  \begin{center}
    \begin{tabular}{|c|}
      \hline
      検知時間 \\
      \hline\hline
       \\\hline
    \end{tabular}
  \end{center}
\end{table}

\begin{table}[htbp]
  \caption{シナリオ2の食事時間}
  \label{tb:scenario_2_meal}
  \begin{center}
    \begin{tabular}{|c||c|}
      \hline
        & 食事時間 \\
      \hline\hline
      被験者C & 2021-01-26 20:45:45 - 21:06:12 \\\hline
    \end{tabular}
  \end{center}
\end{table}

\begin{table}[htbp]
  \caption{シナリオ2の食事検知時間}
  \label{tb:scenario_2_meal_detect}
  \begin{center}
    \begin{tabular}{|c|}
      \hline
      検知時間 \\
      \hline\hline
      2021-01-26 20:48:02.713074 \\\hline
    \end{tabular}
  \end{center}
\end{table}

\subsection{考察}

以上の結果から、通知をすべきでないシナリオ1では通知は行われず、通知がされるべきシナリオ2では、
期待通り通知が行われ、被験者がインスリン摂取忘れをしていることを知らせることができることがわかった。

