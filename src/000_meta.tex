\newif\ifjapanese

\japanesetrue

\ifjapanese
  \documentclass[a4j,11pt]{jreport}
  % \documentclass[a4j,twoside,openright,11pt]{jreport} % 両面印刷の場合。余白を綴じ側に作って右起こし。 >>bindermode<<

  \renewcommand{\bibname}{参考文献}
  \newcommand{\acknowledgmentname}{謝辞}
\else
  \documentclass[a4paper,11pt]{report}
  \newcommand{\acknowledgmentname}{Acknowledgment}
\fi

\usepackage{../styles/thesis}
\usepackage{siunitx}
\usepackage{ascmac}
\usepackage{graphicx}
\usepackage{multirow}
\usepackage{url}
\usepackage{otf}
\usepackage[dvipdfmx]{hyperref}
\usepackage{pxjahyper}
\usepackage{listings}
\usepackage{color}
\renewcommand{\lstlistingname}{ソースコード}

\definecolor{dkgreen}{rgb}{0,0.6,0}
\definecolor{gray}{rgb}{0.5,0.5,0.5}
\definecolor{mauve}{rgb}{0.58,0,0.82}

\lstset{frame=tb,
  language=Python,
  aboveskip=3mm,
  belowskip=3mm,
  showstringspaces=false,
  columns=flexible,
  basicstyle={\small\ttfamily},
  numbers=none,
  numberstyle=\tiny\color{gray},
  keywordstyle=\color{blue},
  commentstyle=\color{dkgreen},
  stringstyle=\color{mauve},
  breaklines=true,
  breakatwhitespace=true,
  tabsize=3
}
\bibliographystyle{jplain}

% \bindermode %バインダー用予約設定 >>bindermode<<

\jclass   {卒業論文}
\jtitle   {加速度センサーを用いた食事検知による、糖尿病患者のインスリン摂取忘れ通知システムの提案}
\juniv    {慶應義塾大学}
\jfaculty {総合政策学部}
\jauthor  {山田 真也}
\jryear   {2}
\jsyear   {2020}
\jkeyword {IoT、糖尿病、加速度センサー}
\jproject {村井・楠本・中村・高汐・バンミーター・植原・三次・中澤・武田 合同研究プロジェクト}
\jdate    {2021年1月}

\eclass   {Bachelor's Thesis}
\etitle   {A Accelerometer Sensor-Based Tool to Alert Diabetic Patients in case of Forgetting to Take Insulin before Meals}
\euniv    {Keio University}
\efaculty {Bachelor of Policy Management}
\eauthor  {Shinya Yamada}
\eyear    {2020}
\ekeyword {IoT, Diabetes, Accelerometer}
\eproject {Murai/Kusumoto/Nakamura/Takashio/Van Meter/Uehara/Mitsugi/Nakazawa/Takeda Labs}
\edate    {January 2021}
