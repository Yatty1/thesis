\begin{jabstract}

現在、生活が豊かな国では、発展した技術により人々は車、電車などの交通手段を得たため座ったまま移動が可能になり、世界的に運動不足になっている。
運動不足は、様々な生活習慣病をもたらす原因の一つになっており、その一つに糖尿病がある。この糖尿病の薬物療法では、1日の中で食前に3回、
就寝前に1回の計4回インスリンを下腹部に注射器で摂取する、4回法が一般的である。しかしながら、患者の中にはこの服用を忘れてしまう人も存在し、
これを忘れると血糖値が下がらず、大切な血管を傷つけてしまう可能性が高まる。現在、この摂取忘れを食事というイベントに即して、受動的に気づくシステムが存在しないため、
本論文はこれを解決することを目指した。そこで、インスリン摂取というイベントの時間と、食事というイベントの時間を作成したデバイスにより取得し、比較することで、
食事前にインスリンが摂取されたかどうかを即時的に確認し、忘れていた場合に通知するという手法を提案した。ここで提案した手法では、インスリン摂取を検知するデバイスはインスリン注射器を模した
ペンの端にボタンを装着することにより達成した。そして、食事の検知は加速度センサーにより卓上の振動を利用するデバイスを制作した。
これを、20代男性3名と20代女性1名で評価を行った結果、食事検知の総検知率は64\%となった。検知ができたのは、1主3菜のメニューや丼のメニューなど、
食器の持ち替えが起こりやすいものであった。しかし検知ができなかったのは、パスタやステーキなど、食器を持ち上げずとも食べることが可能な食事内容の時であった。
そして、食事以外の行動では、書き物、お茶を飲む、マウスを使ってネットサーフィンが12.5\%、20\%、20\%という誤検知率となった。
また、通知の評価では、インスリンを忘れて通知を行うべきシナリオで、通知があった割合は66.67\%となったが、食事検知が行われた場合の通知成功率は100\%であった。
本論文は、食事検知の対象が限られてしまうなど、限定的なものになってしまったものの卓上の振動のみで食事という特定のイベントを検知し、活用できる可能性を示した。
そして、今回提案した手法は、インスリン摂取を忘れてしまいがちな糖尿病患者の一助となれる可能性があると考えられる。

\end{jabstract}
