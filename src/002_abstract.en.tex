\begin{eabstract}

  Nowadays, in countries where life is affluent, developed technology has enabled people to move around in a sitting position due to the availability of cars, trains and other means of transportation, resulting in a worldwide lack of exercise.
  Lack of exercise is one of the causes of various lifestyle-related diseases, one of which is diabetes. In the drug therapy of diabetes, insulin is generally taken four times a day, three times before meals and once before bedtime, using a syringe in the lower abdomen.
  However, some patients may forget to take these doses. Forgetting to take this dose increases the possibility that the blood glucose level will not drop and that blood vessels will be damaged.
  Currently, there is no system to notify patients of this missed dose in the context of a meal event automatically.
  This paper aimed to solve this problem by comparing the time of the event of insulin intake with the time of the event of a meal using a device that we created.
  By comparing the time of the event of insulin intake and the time of the event of meal, we proposed a method to immediately check if insulin was taken before the meal and to notify the user if he or she forgot it.
  In the proposed method, the device to detect insulin intake is a pen with a button attached to the end of the pen that resembles an insulin syringe.
  In the method proposed here, the device to detect insulin intake is a button attached to the end of a pen that mimics an insulin syringe. We also created a device that uses tabletop vibration with an accelerometer to detect meals.
  This was evaluated with the cooperation of three subjects, and it was confirmed that the device could detect meals. Finally, with the cooperation of one subject, we were able to confirm that a notification was given when the subject ate without taking insulin.
  This paper has shown the possibility of using tabletop vibration alone to detect and identify specific events such as meals.
  And also the proposed method has the potential to help diabetic patients who tend to forget to take insulin.

\end{eabstract}
