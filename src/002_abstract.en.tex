\begin{eabstract}

  Nowadays, in countries where life is affluent, developed technology has enabled people to move around in a sitting position due to the availability of cars, trains and other means of transportation, resulting in a worldwide lack of exercise.
  Lack of exercise is one of the causes of various lifestyle-related diseases, one of which is diabetes. The drug therapy for diabetes is to take insulin three times a day before meals and once before bedtime. However, some patients may forget to take these doses.
  Forgetting to take this dose, however, increases the possibility that the blood glucose level will not drop, and blood vessels will be damaged. Currently, there is no system to notify this missed dose in the context of a meal event automatically.
  In this paper, we aim to solve this problem by comparing the time of the event of insulin intake with the time of the event of a meal using a device that we created.
  By comparing the time of the event of insulin intake and the time of the event of meal, we proposed a method to immediately check if insulin was taken before the meal and to notify the user if he or she forgot.
  In the method proposed here, the device to detect insulin intake is a button attached to the end of a pen that mimics an insulin syringe. For meal detection, we created a device that uses tabletop vibration with an accelerometer.
  This was evaluated by three males in their twenties and one female in her twenties, resulting in a total detection rate of 64\% for meal detection. The detection rate was 64\% for meals with one or three main dishes and rice bowls.
  However, those that failed to detect were those that had a tendency to change dishes. However, detection was not possible for meals such as pasta and steak, which can be eaten without lifting the dishes.
  As for non-meal behaviors, writing, drinking tea, and surfing the Internet with a mouse had false positive rates of 12.5\%, 20\%, and 20\%, respectively.
  In addition, in the evaluation of notification, in the scenario where insulin was forgotten and notification should be made, the percentage of notification was 66.67\%, but the success rate of notification when meal detection was used was 100%.
  This paper has shown the possibility of detecting the specific event of a meal using only tabletop vibration, although the target of meal detection was limited.
  The proposed method may help diabetic patients who tend to forget to take their insulin.

\end{eabstract}
