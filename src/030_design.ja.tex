\chapter{提案手法、アプローチ}
\label{chap:design}

本章では、本研究の目的を達成するための手法とその設計を示す。

\section{全体の構成}
本研究は、大きく分けて二つの構成要素に分けられる。

\begin{enumerate}
  \item インスリンに装着するデバイス
  \item 被験者の食事を検知する加速度センサーデバイス
\end{enumerate}

1のインスリンデバイスはインスリン摂取時間を検知するために用いる。
2の加速度センサーデバイスは、食卓上に設置することで被験者の食事開始時間を検知する。

\ref{fig:insulin_design}にその構成図を示す。

\begin{figure}
  \caption{インスリン摂取忘れ検知のシステム構成図}
  \label{fig:insulin_design}
  \begin{center}
    % \includegraphics[bb=0 0 1000 530,width=15cm]{assets/}
  \end{center}
\end{figure}

\section{検知方法}

\ref{fig:insulin_design}における、1,2のモジュールによって収集された検知時間データを下に、被験者が食前にインスリンを摂取できているかを求めることができ、
その結果に応じて通知を行うことができる。
検知にはそれぞれの実検知時間を使用し、最終インスリン摂取時間と最終食事検知時間を比較し、その時間差が一定時間以内であれば、
インスリンは正しいタイミングで摂取されたと判断する。
その差が一定時間以上であれば、インスリン摂取は行われなかったとし、患者に対して通知を行う。

